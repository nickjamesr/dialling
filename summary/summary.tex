\documentclass[a4paper]{article}

\setlength{\parindent}{0mm}
\setlength{\parskip}{2mm}

\newcommand{\by}{\times}
\newcommand{\of}[1]{\left(#1\right)}
\newcommand{\bigo}[1]{O\of{#1}}

\usepackage{geometry}
\title{Direct Dialling}
\author{Nick Russell}

\begin{document}
\maketitle
This project concerns how to dial a Haar unitary random matrix in a Reck scheme
without having to first generate the unitary explicitly (e.g.\ by Gramm-Schmidt)
and then convert to Reck scheme parameters. Instead, the \( N^{2} \) parameters
required are chosen from appropriate distributions such that the effect of the
Reck scheme is a Haar unitary random matrix. I have found the forms of these
distributions and proved via a coordinate transformation that the do in fact
result in a random unitary, as required. Moreover, the time taken to generate
the Reck scheme parameters is \(\bigo{N^{2}}\), while the Gramm-Schmidt process
takes \(\bigo{N^{3}}\). My scheme is thus a more efficient way of generating the
Reck scheme, but note that it still takes \(\bigo{N^{3}}\) time to generate the
explicit form of the unitary matrix.

Unfortunately, work very similar to this (though in more abstract terms) has
already been published by Spengler et al. I must finish writing up my version,
then decide if there is sufficient original content to attempt to publish it.

The c

I have also done some preliminary numerical work on a different decomposition
using binary branching, rather than the `cascade' form of the Reck scheme. This
results in different probability distributions, which are all centered on 0.5,
and approximate Gaussians near to the root of the (binary) tree. The
approximation to a Gaussian is very good. This may prove more useful for
fabrication of \(d\)-way splitters, because it eliminates the need for
beamsplitters of very low reflectivity. However, to realise a full unitary in
this way requires many crossings of waveguides, which is hard to fabricate
(though a 3D approach could eliminate some of these problems). Finally, path
length matching will require some careful consideration.

Another use for the Reck decomposition is for generating truncations of
Haar unitaries directly, without having to generate the entire unitary matrix.
Say I want a \(p \by p\) matrix that is the truncation of an \(m \by m\) Haar
unitary, and \(m = \bigo{p^{2}}\), I would traditionally have to generate an \(m
\by m\) unitary, then take (say) the top left corner. I could reduce the work
somewhat by just generating the first \(p\) columns, but I am still throwing
away a lot of the matrix. Using the Reck scheme I can use a `diamond' shape of
beamsplitters, chosen from the top \(2p\) modes of the whole scheme, and
generate the unitary corresponding to this. The bottom \(p\) modes are just
loss, but the top left \(p \by p\) matrix corresponds to a truncation of an \(m
\by m\) Haar unitary.



\end{document}

