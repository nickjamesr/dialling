\documentclass[a4paper,11pt]{article}

\usepackage{graphicx}
\usepackage{geometry}
\usepackage{amsmath}

\setlength{\parindent}{0pt}
\setlength{\parskip}{4mm}

\newcommand{\mat}[1]{\mathrm{#1}}

\begin{document}
A 4 \(\times\) 4 Reck scheme:

\begin{figure}[h]
  \setlength{\unitlength}{1cm}
  \begin{picture}(14,5)
    \put(1,1){\includegraphics[width=12cm]{reck_4x4}}
    % Labels for the subunitaries
    \put(1.0,4.5){\( \mat{U}_{3} \)}
    \put(2.5,4.5){\( \mat{U}_{2} \)}
    \put(5.3,4.5){\( \mat{U}_{1} \)}
    \put(10,4.5){\( \mat{U}_{0} \)}
    % Labels for the phase shifts
    \put(12.2, 2.5){\( \phi_{01} \)}
    \put(10.4, 3.4){\( \phi_{02} \)}
    \put(8.6, 4.3){\( \phi_{03} \)}
    \put(6.6, 2.5){\( \phi_{12} \)}
    \put(5.0, 3.4){\( \phi_{13} \)}
    \put(3.1, 2.5){\( \phi_{23} \)}
    % Labels for the reflectivities
    \put(11.5, 0.8){\( r_{01} \)}
    \put(9.7, 1.7){\( r_{02} \)}
    \put(8.0, 2.6){\( r_{03} \)}
    \put(6.1, 0.8){\( r_{12} \)}
    \put(4.4, 1.7){\( r_{13} \)}
    \put(2.5, 0.8){\( r_{23} \)}
    % Labels for the input phases
    \put(1.0, 3.5){\( \alpha_{0} \)}
    \put(1.0, 2.6){\( \alpha_{1} \)}
    \put(1.0, 1.7){\( \alpha_{2} \)}
    \put(1.0, 0.8){\( \alpha_{3} \)}
  \end{picture}
\end{figure}
The programme converts a unitary into a representation as a Reck scheme. This is
stored as a matrix of parameters:
\begin{equation*}
  U \rightarrow \begin{pmatrix}
    R_{00} & R_{01} & R_{02} & R_{03} \\
    R_{10} & R_{11} & R_{12} & R_{13} \\
    R_{20} & R_{21} & R_{22} & R_{23} \\
    R_{30} & R_{31} & R_{32} & R_{33} \end{pmatrix}
\end{equation*}
The upper triangle contains the reflectivities, the lower triangle the phase
shifts, except for the input phases, which are on the diagonal, i.e.
\begin{align*}
  r_{ij} &= R_{ij} \\
  \phi_{ij} &= R_{ji} \\
  \alpha_{i} &= R_{ii}
\end{align*}

\end{document}
